\documentclass[../../main.tex]{subfiles}
\begin{document}
\tracingall

We have presented a combined algorithm based on the two algorithms regional time stepping and two-scale simulation. Seeing as we can run complete simulations and that the visual results do not differ noticeably from the base algorithms, the answer to our first research question is: it is possible to construct a combined algorithm. 

However, our implementation of the two-scale algorithm seems to give incorrect results and since our combined algorithm utilizes the same implementation we can not be sure that our results are correct. Although we see a slight speed up, we can not with certainty answer our second research question as the results undoubtedly would change should the two-scale algorithm give better results. 

To conclude, our combined method has potential, but it is restricted our two-scale implementation. In addition, the speed up we already have makes up for the algorithms relative complexity compared to RTS and two-scale simulation. 


%%%%%%%%
\section{Future work}
%%%%%%%%

One continuation would be to try and reproduce Solenthaler and Gross' results with the two-scale implementation. This might be as simple as building better suited scenes or implementing support for a resolution factor of 4, or it might require a whole re-implementation of the algorithm. Regardless of the two-scale implementation, the algorithm presented in this paper could also be expanded using the GPU rather than the CPU. As mentioned by \citet{goswami2014regional} the blocks can be treated as parallelization units for computing the physics of particles within, as also implemented by \citet{goswami2010interactive}.

To further explore the possibility to perform more complex calculations for determining the high resolution area for particles in L could be interesting. In our algorithm we manually determine, prior to the simulation, one single 3-dimensional cuboid which we use to perform simple location based queries against. However, as suggested by \citet{solenthaler2011two}, exploring the possibilities with either surface detection or using a view frustum to define high resolution areas could be an interesting next step. 

There is another point of view that could be explored. Rather than combining the two methods as shown in this paper it could also be interesting to explore the possibility to take advantage of differently sized particles in different regions. For instance letting $\Re_1$ and $\Re_2$ be H-regions or even having one particle size per region level. 

We would also like to run our tests several more times and perform statistical analyses on the results. This way we could establish the validity of our current results. 

\end{document}