\documentclass[../../main.tex]{subfiles}
\begin{document}


% Wrap uo
% Results
% Answer research questionns
% Future work


\textit{Summary. What have we done, what did we acheive, why did we acheive this. What factors was important, specific conditions such as language, OS, frameworke and machine.}
We have combined two different techniques that simulates fluids, which both focus on reducing simulation times, and have successfully utilized both techniques and reduced simulation times even further.

This allows for a fewer amount of particles to represent a substantially larger area than would be possible without resizing them. Thus the simulation does not need as many particles which reduces the computation time. The region of higher resolution can be set where the details of the simulation is more important, like the surface, the view frustum, or where obstacles are.


%%%%%%%%
\section{Future work}
%%%%%%%%
\textit{How can this be expanded in the future? What implications does this have on existing work? Could it be combined with any already technique?}
Surface detection thingy. 
Usage of general purpose graphical units (GPU) could further be used to speedup. 
We have presented a method where we effectively split a fluid simulation into two distinct parts, the low resolution set L and the subset of L that represents the high resolution part H. The low resolution uses particles with a smaller number or particles.
Moreover, we continue by breaking these sets into three-dimensional regions, also called blocks, in such a way that each particle exist in exactly one block. Each block then calculates the smallest time-step required based on the attributes of every particle in the block. This is an attempt to ensure the highest possible time-step for that block, and thus its particles. 
%This is done twice per frame, once for the high resolution and once for the low resolution set.

The method presented in this paper could also be expanded using the Graphic Processing Unit(GPU) rather than the Central Processing Unit(CPU). As mentioned in \cite{goswami2014regional} the blocks can be treated as parallelization units for computing the physics of particles within, as also implemented in \cite{goswami2010interactive}.

There is another point of view that could be explored. Rather than combining the two methods as shown in this paper it could also be interesting to explore the possibility to take advantage of differently sized particles in different regions.

To further explore the possibility to perform more complex calculations for determining the high resolution area for particles in L could be interesting. In this paper we manually determines, prior to the simulation, one single 3-dimensional cuboid which we use to performs simple location based queries against. Taking this to the next level and explore the possibility with either surface detection or using a view frustum to define high resolution areas could be an interesting next step. 

Another interesting idea would be to dynamically set the H area to the R1 and R2 since that is probably where the most action will be, but that will introduce very many boundary regions.

%%%%%%%%
\section{Lessons learned}
%%%%%%%%
Just because one have access to someone else codebase, does not mean that the workload decreases very much. Simulating fluids with multi-resolution particles is difficult. CFL condition is not very straightforward. 

\end{document}