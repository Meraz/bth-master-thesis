\documentclass[../main.tex]{subfiles}
\begin{document}

\textit{This section should not list many references}
\textit{Old text (Background)}
\hspace{1pt}

%%%%%%%%
\section{Fluid simulation 101 Background}
%%%%%%%%
\textit{Background. Basic fluid simulation, what do we use it for, what does it mean, method foundation (Lagrangian, Eulerian). What problems do these methods create and how does that affect the use. Other fundamental methods to solve these problems. And once again, problems these methods creates. Computational expensive. }
In the field of computer grapichs, fluid simulation is a tool to produce realistic animation of fluids such as water and fluid. A simulation emulates the motion of fluids with the use of either Euler equations or Navier-Stokes equations, or any simpler version of them. Multiple approaches exists with varying positive key aspects and range in complexity from time consuming computational expensive high quality animations, simpler real-time particle systems, or Fourier synthesis of water surface wave. 

Realistic fluid is often sought after in computer graphics, however, multiple techniques exist with a variety of trade-offs, there are no go to technique that always work. Approaches   

Eulerian or grid-based methods are a common choice for simulating fluids in the industry due to higher coherence with the ground reality. On the other hand, Lagrangian methods like Smooth Particles Hydrodynamics (SPH) offer advantages on creating small scale features like droplets and splashes. Further they conserve mass implicitly and are simple to implement.


%%%%%%%%
\section{SPH}
%%%%%%%%
\textit{Might not be a subsection. Mention basic SPH: particle spacing, particle count, timestep, neighborhood search. Mention problems with lowering the time step, altering the particle count and spacing, and neighborhood search. }


%%%%%%%%
\section{Methods for increasing performance}
%%%%%%%%
\textit{Continuation on the SPH. Mention solutions for earlier mentioned problems. Mention }



\end{document}